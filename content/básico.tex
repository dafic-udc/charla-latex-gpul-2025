% Comandos básicos

% Crear portada (Campos básicos)

\section{Comandos básicos}

\begin{frame}[fragile]
\frametitle{Comandos básicos: portada}

Opción fácil $\rightarrow$ \verb|\maketitle|
\begin{itemize}
     \item \verb|\title{Titulo documento}|
     \item \verb|\author{Daniel Feito \& Hugo Herrador \& Lúa Rico}|
     \item \verb|\date{March 2025}|
\end{itemize}

\vspace{0.5cm}

Opción avanzada $\rightarrow$ \verb|\begin{titlepage}|


\end{frame}

% Section, subsection
\begin{frame}[fragile]
\frametitle{Comandos básicos: secciones, índice y páginas}
Crear secciones y subsecciones
\begin{itemize}
    \item \verb|\section{}|
    \item \verb|\subsection{}|
    \item \verb|\subsubsection{}| 
\end{itemize}

\vspace{0.5cm}

Crear el índice (se actualiza automáticamente) $\rightarrow$ \verb|\tableofcontents|

\vspace{0.5cm}

Crear una nueva página  $\rightarrow$ \verb|\newpage|

\end{frame}

% Cargar paquetes

\begin{frame}[fragile]
\frametitle{Comandos básicos: cargar paquetes}
Normalmente se hace justo después de definir el tipo de documento. 

\vspace{0.5cm}

\verb|\usepackage{nombre_paquete}|
\end{frame}

% Negrita, cursiva, colores en texto
\begin{frame}[fragile]
\frametitle{Comandos básicos: negrita, cursiva y colores}
\begin{itemize}
    \item \textbf{Negrita} $\rightarrow$ \verb|\textbf{Negrita}|
    \item \textit{Cursiva} $\rightarrow$ \verb|\textit{Cursiva}|
    \begin{itemize}
        \item [*] Otra forma de escribir cursiva es con el comando \verb|\emph{}| (`enfatizar' texto)
        \item [*] La diferencia es que si se usa 2 veces \verb|\emph{}|, el texto vuelve a escribirse normal, pero si se usa 2 veces \verb|\textit{}| el texto se queda en cursiva
    \end{itemize}
    \item \textcolor{red}{Colores} $\rightarrow$ \verb|\textcolor{red}{Colores}| (necesario el paquete \verb|\usepackage{xcolor}|)
\end{itemize}

\end{frame}

