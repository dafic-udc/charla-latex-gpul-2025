\section{Listas}

\begin{frame}[fragile]
\frametitle{Listas: Tipos y formato}

Tipos de listas:
\begin{itemize}
 \item Listas ordenadas $\rightarrow$ \verb|enumerate|
 \item Listas no ordenadas $\rightarrow$ \verb|itemize|
 \item Listas de descripciones $\rightarrow$ \verb|description|
\end{itemize}

\vspace{0.5cm}

Cada elemento se separa con \verb|\item| .

\vspace{0.5cm}

Las listas pueden anidarse (aparecer una dentro de otra), incluso entre tipos distintos.
\end{frame}


\begin{frame}[fragile]
\frametitle{Listas: Ejemplo de lista ordenada}
\begin{columns}
\column{0.45\textwidth}

    \begin{enumerate}
    \item Item A
    \item Item B
        \begin{enumerate}
            \item Item B.1
            \item Item B.2
        \end{enumerate}
    \item Item C
    \item Item D
    \end{enumerate}

\column{0.45\textwidth}
\begin{semiverbatim}
\\begin\{enumerate\}
    \\item Item A
    \\item Item B
        \\begin\{enumerate\}
            \\item Item B.1
            \\item Item B.2
        \\end\{enumerate\}
    \\item Item C
    \\item Item D
\\end\{enumerate\}
\end{semiverbatim}
\end{columns}
\end{frame}


\begin{frame}[fragile]
\frametitle{Listas: Ejemplo de lista no ordenada}
\begin{columns}
\column{0.45\textwidth}

    \begin{itemize}
    \item Item A
    \item Item B
        \begin{itemize}
            \item Item B.1
            \item Item B.2
        \end{itemize}
    \item Item C
    \item Item D
    \end{itemize}

\column{0.45\textwidth}
\begin{semiverbatim}
\\begin\{itemize\}
    \\item Item A
    \\item Item B
        \\begin\{itemize\}
            \\item Item B.1
            \\item Item B.2
        \\end\{itemize\}
    \\item Item C
    \\item Item D
\\end\{itemize\}
\end{semiverbatim}
\end{columns}
\end{frame}

\begin{frame}[fragile]
\frametitle{Listas: Ejemplo de lista de descripciones}
\begin{description}
   \item[Término A] Descripción súper ingeniosa aquí.
   \item[Término B] Otra descripción creativa aquí.
\end{description}

\vspace{0.5cm}

\begin{semiverbatim}
\\begin\{description\}
   \\item[Término A] Descripción súper ingeniosa aquí.
   \\item[Término B] Otra descripción creativa aquí.
\\end\{description\}
\end{semiverbatim}
\end{frame}

\begin{frame}[fragile]
\frametitle{Listas: Estilos}
Estilos de las listas:
\begin{itemize}
    \item Las listas ordenadas por defecto aparecen con números.
    \item Las listas no ordenadas por defecto aparecen con puntos.
    \item El estilo de las listas anidadas puede cambiar en cada nivel de profundidad.
    \item El decorador de cada \verb|\item| puede personalizarse indicándolo entre corchetes:
\end{itemize}

\begin{columns}
\column{0.4\textwidth}

\begin{itemize}
    \item[-] Ejemplo con \texttt{[-]}.
    \item[!] Ejemplo con \texttt{[!]}.
    \item[XD] Ejemplo con \texttt{[XD]}.
    \item[] Ejemplo con \texttt{[]}.
\end{itemize}

\column{0.65\textwidth}
\begin{semiverbatim}
\\begin\{itemize\}
    \\item[-] Ejemplo con \\texttt{[-]}.
    \\item[!] Ejemplo con \\texttt{[!]}.
    \\item[XD] Ejemplo con \\texttt{[XD]}.
    \\item[] Ejemplo con \\texttt{[]}.
\\end\{itemize\}
\end{semiverbatim}
\end{columns}

\end{frame}
