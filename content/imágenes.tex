\section{Imágenes}

\begin{frame}[fragile]{Imágenes: \texttt{includegraphics}}

\verb|\includegraphics| permite insertar una imagen mediante su ruta realtiva al fichero \texttt{.tex} principal. 

\vspace{0.5cm}

También permite establecer su tamaño y rotación. Sin embargo, es muy limitado en cuanto al control de la imagen.

\includegraphics[width=0.5\textwidth]{images/logo_gpul.png}

\begin{verbatim}
\includegraphics[width=0.5\textwidth]{images/logo_gpul.png}
\end{verbatim}
\end{frame}


\begin{frame}[fragile]{Imágenes: \texttt{figure}}

El entorno \texttt{figure} sirve para mostrar las imágenes como elementos flotantes, facilita controlar sus posiciones y permite opciones adicionales.

\begin{figure}[h]
\includegraphics[width=0.3\textwidth]{images/logo_gpul.png}
\caption{El chulísimo logo del GPUL.}
\label{img:logo_gpul}
\end{figure}

\begin{verbatim}
\begin{figure}[h]
\includegraphics[width=0.3\textwidth]{images/logo_gpul.png}
\caption{El chulísimo logo del GPUL.}
\label{img:logo_gpul}
\end{figure}
\end{verbatim}

\end{frame}
