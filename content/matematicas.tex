\section{Matemáticas}

\begin{frame}[fragile]
    Calculemos la siguiente integral definida:
    \begin{equation}
        \int_{- \infty}^{+ \infty}{e^{-x^2}dx}
    \end{equation}
    \begin{verbatim}
    \begin{equation}
        \int_{- \infty}^{+ \infty}{e^{-x^2}dx}
    \end{equation}
    \end{verbatim}
    \transparent{0.3}
    Otra opción: $\int_{- \infty}^{+ \infty}{e^{-x^2}dx}$
    \begin{verbatim}
        $\int_{- \infty}{+ \infty}{e^{-x^2}dx}$
    \end{verbatim}
    Tercera opción: $$\int_{- \infty}^{+ \infty}{e^{-x^2}dx}$$
    \begin{verbatim}
        $$\int_{- \infty}{+ \infty}{e^{-x^2}dx}$$
    \end{verbatim}
\end{frame}

\begin{frame}[fragile]
    Calculemos la siguiente integral definida:
    \transparent{0.3}
    \begin{equation*}
        \int_{- \infty}^{+ \infty}{e^{-x^2}dx}
    \end{equation*}
    \begin{verbatim}
    \begin{equation}
        \int_{- \infty}^{+ \infty}{e^{-x^2}dx}
    \end{equation}
    \end{verbatim}
    \transparent{1}
    Otra opción: $\int_{- \infty}^{+ \infty}{e^{-x^2}dx}$
    \begin{verbatim}
        $\int_{- \infty}{+ \infty}{e^{-x^2}dx}$
    \end{verbatim}
    \transparent{0.3}
    Tercera opción: $$\int_{- \infty}^{+ \infty}{e^{-x^2}dx}$$
    \begin{verbatim}
        $$\int_{- \infty}{+ \infty}{e^{-x^2}dx}$$
    \end{verbatim}
\end{frame}

\begin{frame}[fragile]
    Calculemos la siguiente integral definida:
    \transparent{0.3}
    \begin{equation*}
        \int_{- \infty}^{+ \infty}{e^{-x^2}dx}
    \end{equation*}
    \begin{verbatim}
    \begin{equation}
        \int_{- \infty}^{+ \infty}{e^{-x^2}dx}
    \end{equation}
    \end{verbatim}
    \transparent{0.3}
    Otra opción: $\int_{- \infty}^{+ \infty}{e^{-x^2}dx}$
    \begin{verbatim}
        $\int_{- \infty}{+ \infty}{e^{-x^2}dx}$
    \end{verbatim}
    \transparent{1}
    Tercera opción: $$\int_{- \infty}^{+ \infty}{e^{-x^2}dx}$$
    \begin{verbatim}
        $$\int_{- \infty}{+ \infty}{e^{-x^2}dx}$$
    \end{verbatim}
\end{frame}

\begin{frame}[fragile]
    \begin{center}
    Sea $\mathcal{I} = \int_{- \infty}^{+ \infty}{e^{-x^2}dx}$
    \end{center}
    \transparent{0}
    \begin{align*}
        \mathcal{I}^2 &= \left ( \int_{- \infty}^{+ \infty}{e^{-x^2}dx} \right )^2 \\
        &= \left ( \int_{- \infty}^{+ \infty}{e^{-x^2}dx} \right ) \cdot \left ( \int_{- \infty}^{+ \infty}{e^{-y^2}dy} \right ) \\
        &= \left ( \int_{- \infty}^{+ \infty}{e^{-(x^2 + y^2)}dx} \right )
    \end{align*}
    \begin{center}
    ¿Qué podemos hacer ahora?
    \end{center}
    \begin{center}
    ¡Un cambio a polares!
    \end{center}
    $$
    \begin{cases}
        x = r \cdot \cos{\theta} \\
        y = r \cdot \sin{\theta}
    \end{cases}
    $$
\end{frame}

\begin{frame}[fragile]
    \begin{center}
    Sea $\mathcal{I} = \int_{- \infty}^{+ \infty}{e^{-x^2}dx}$
    \end{center}
    \begin{align*}
        \mathcal{I}^2 &= \left ( \int_{- \infty}^{+ \infty}{e^{-x^2}dx} \right )^2 \\
        &= \left ( \int_{- \infty}^{+ \infty}{e^{-x^2}dx} \right ) \cdot \left ( \int_{- \infty}^{+ \infty}{e^{-y^2}dy} \right ) \\
        &= \left ( \int_{- \infty}^{+ \infty}{e^{-(x^2 + y^2)}dx} \right )
    \end{align*}
    \transparent{0}
    \begin{center}
    ¿Qué podemos hacer ahora?
    \end{center}

    \begin{center}
    ¡Transformar a polares!
    \end{center}
    $$
    \begin{cases}
        x = r \cdot \cos{\theta} \\
        y = r \cdot \sin{\theta}
    \end{cases}
    $$
\end{frame}

\begin{frame}[fragile]
    \begin{center}
    Sea $\mathcal{I} = \int_{- \infty}^{+ \infty}{e^{-x^2}dx}$
    \end{center}
    \begin{align*}
        \mathcal{I}^2 &= \left ( \int_{- \infty}^{+ \infty}{e^{-x^2}dx} \right )^2 \\
        &= \left ( \int_{- \infty}^{+ \infty}{e^{-x^2}dx} \right ) \cdot \left ( \int_{- \infty}^{+ \infty}{e^{-y^2}dy} \right ) \\
        &= \left ( \int_{- \infty}^{+ \infty}{e^{-(x^2 + y^2)}dx} \right )
    \end{align*}
    \begin{center}
    ¿Qué podemos hacer ahora?
    \end{center}
    \transparent{0}
    \begin{center}
    ¡Transformar a polares!
    \end{center}
    $$
    \begin{cases}
        x = r \cdot \cos{\theta} \\
        y = r \cdot \sin{\theta}
    \end{cases}
    $$
\end{frame}

\begin{frame}[fragile]
    \begin{center}
    Sea $\mathcal{I} = \int_{- \infty}^{+ \infty}{e^{-x^2}dx}$
    \end{center}
    \begin{align*}
        \mathcal{I}^2 &= \left ( \int_{- \infty}^{+ \infty}{e^{-x^2}dx} \right )^2 \\
        &= \left ( \int_{- \infty}^{+ \infty}{e^{-x^2}dx} \right ) \cdot \left ( \int_{- \infty}^{+ \infty}{e^{-y^2}dy} \right ) \\
        &= \left ( \int_{- \infty}^{+ \infty}{e^{-(x^2 + y^2)}dx} \right )
    \end{align*}
    \begin{center}
    ¿Qué podemos hacer ahora?
    \end{center}

    \begin{center}
    ¡Transformar a polares!
    \end{center}
    $$
    \begin{cases}
        x = r \cdot \cos{\theta} \\
        y = r \cdot \sin{\theta}
    \end{cases}
    $$
\end{frame}

\begin{frame}[fragile]
    \begin{align*}
        \mathcal{I}^2 &= \int_{0}^{2 \pi}\int_0^{+ \infty} \begin{vmatrix} \frac{\partial x}{\partial r} && \frac{\partial x}{\partial \theta} \\ \frac{\partial y}{\partial r} && \frac{\partial y}{\partial \theta}\end{vmatrix} \cdot e^{-r^2} dr d\theta \\
        &= \int_{0}^{2 \pi}\int_0^{+ \infty} \begin{vmatrix} \cos \theta && -r \sin \theta \\ \sin \theta && r \cos \theta \end{vmatrix} \cdot e^{-r^2} dr d\theta \\
        &= \int_{0}^{2 \pi}\int_0^{+ \infty} r \cdot e^{-r^2} dr d\theta \\
        &= -\frac{1}{2} \cdot \int_{0}^{2 \pi}\int_0^{+ \infty} r \cdot -2r \cdot e^{-r^2} dr d\theta \\
        &= -\frac{1}{2} \cdot \int_{0}^{2 \pi} \left . e^{-r^2} \right |_{0}^{+ \infty} d\theta\\
        &= -\frac{1}{2} \cdot \int_{0}^{2 \pi} - d\theta \\
        &= \frac{1}{2} \cdot 2\pi \\
        &= \pi \implies \boxed{\mathcal{I} = \sqrt{\pi}}
    \end{align*}
\end{frame}

\begin{frame}[fragile]
\frametitle{Ajustar tamaño de delimitadores}
    \begin{center}
    \begin{equation*}
    \mathcal{I}^2 = \left ( \int_{- \infty}^{+ \infty}{e^{-x^2}dx} \right )^2
    \end{equation*}
    \begin{verbatim}
    \mathcal{I}^2 =
    \left ( \int_{- \infty}^{+ \infty}
    {e^{-x^2}dx} \right )^2
    \end{verbatim}
    \end{center}
    \begin{itemize}
        \item{\verb|\left| y \verb|\right| se usan para ajustar el tamaño de los delimitadores} 
        \item{Podemos usar . si sólo necesitamos uno de los dos lados}
        \item{Para los subíndices y superíndices usamos \_ y \^{}, respectivamente}
    \end{itemize}
\end{frame}


\begin{frame}[fragile]
\frametitle{Alinear fórmulas respecto del símbolo =}

\begin{align*}
    \mathcal{I}^2 &= \left ( \int_{- \infty}^{+ \infty}{e^{-x^2}dx} \right )^2 \\
    &= \left ( \int_{- \infty}^{+ \infty}{e^{-x^2}dx} \right ) \cdot \left ( \int_{- \infty}^{+ \infty}{e^{-y^2}dy} \right )
\end{align*}

\begin{verbatim}
\begin{align*}
    \mathcal{I}^2 &= \left ( \int {e^{-x^2}dx} \right )^2 \\
    &= \left ( \int {e^{-x^2}dx} \right ) \cdot
       \left ( \int{e^{-y^2}dy} \right )
\begin{align}
\end{verbatim}
\end{frame}

\begin{frame}[fragile]
    \frametitle{Símbolos varios}
        \begin{align*}
            \lim_{x\to x_{0}}f(x) = l :&\iff \Big[  \forall\varepsilon>0, \exists\delta>0 \Big/ 0<\left|x-x_0\right|<\delta \\
            &\implies \left|f(x)-l\right| <\varepsilon \Big]
        \end{align*}
        \begin{itemize}
            \item{Letras griegas (\textbackslash varepsilon, \textbackslash delta...)}
            \item{Operadores lógicos (\textbackslash land, \textbackslash lor...)}
            \item{Implicaciones (\textbackslash implies, \textbackslash iff)}
            \item{Cuantificador existencial y universal (\textbackslash forall \textbackslash exists)}
            \item{Operadores de conjuntos (\textbackslash cup, \textbackslash cap, \textbackslash subset, \textbackslash in, \textbackslash notin...)}
        \end{itemize}
    \transparent{0}
    \center{\textbf{\textcolor{red}{RETO}: ¿NOS ATREVEMOS A COPIAR LA DEFINICIÓN DE LÍMITE?}}
\end{frame}

\begin{frame}[fragile]
    \frametitle{Símbolos varios}
        \begin{align*}
            \lim_{x\to x_{0}}f(x) = l :&\iff \Big[  \forall\varepsilon>0, \exists\delta>0 \Big/ 0<\left|x-x_0\right|<\delta \\
            &\implies \left|f(x)-l\right| <\varepsilon \Big]
        \end{align*}
        \begin{itemize}
            \item{Letras griegas (\textbackslash varepsilon, \textbackslash delta...)}
            \item{Operadores lógicos (\textbackslash land, \textbackslash lor...)}
            \item{Implicaciones (\textbackslash implies, \textbackslash iff)}
            \item{Cuantificador existencial y universal (\textbackslash forall \textbackslash exists)}
            \item{Operadores de conjuntos (\textbackslash cup, \textbackslash cap, \textbackslash subset, \textbackslash in, \textbackslash notin...)}
        \end{itemize}
    \center{\textbf{\textcolor{red}{RETO}: ¿NOS ATREVEMOS A COPIAR LA DEFINICIÓN DE LÍMITE?}}
\end{frame}
